% 代码排版
\usepackage{listingsutf8}
% \usepackage{listings}
\usepackage{xcolor}
\usepackage{fontspec}
\usepackage{color}

\definecolor{dkgreen}{rgb}{0,0.6,0}
\definecolor{gray}{rgb}{0.5,0.5,0.5}
\definecolor{mauve}{rgb}{0.58,0,0.82}

% \lstset{
%     basicstyle          =   \sffamily,          % 基本代码风格
%     keywordstyle        =   \bfseries,          % 关键字风格
%     commentstyle        =   \rmfamily\itshape,  % 注释的风格,斜体
%     stringstyle         =   \ttfamily,  % 字符串风格
%     flexiblecolumns,                % 别问为什么,加上这个
%     numbers             =   left,   % 行号的位置在左边
%     showspaces          =   false,  % 是否显示空格,显示了有点乱,所以不现实了
%     numberstyle         =   \zihao{-5}\ttfamily,    % 行号的样式,小五号,tt等宽字体
%     showstringspaces    =   false,
%     captionpos          =   t,      % 这段代码的名字所呈现的位置,t指的是top上面
%     frame               =   lrtb,   % 显示边框
% }

\lstset{ %
    backgroundcolor =   \color{white},    % choose the background color
    basicstyle      =   \footnotesize\ttfamily, % size of fonts used for the code
    columns         =   fullflexible,
    tabsize         =   4,
    breaklines      =   true,             % automatic line breaking only at whitespace
    captionpos      =   b,                % sets the caption-position to bottom
    commentstyle    =   \color{mygreen},  % comment style
    escapeinside    =   {\%*}{*)},        % if you want to add LaTeX within your code
    keywordstyle    =   \color{blue},     % keyword style
    stringstyle     =   \color{mymauve}\ttfamily, % string literal style
    frame           =   single,
    rulesepcolor    =   \color{red!20!green!20!blue!20},
    identifierstyle =   \color{red},
    language        =   c++,
}

\lstdefinestyle{myPython}{
	frame               =   l,
	language            =   Python,
	aboveskip           =   3mm,
	belowskip           =   3mm,
	showstringspaces    =   false,
	columns             =   flexible,
	numberstyle         =   \small\color{red},
	basicstyle          =   {\small\ttfamily},
	keywordstyle        =   \color{blue},
	commentstyle        =   \color{dkgreen},
	stringstyle         =   \color{mauve},
	breaklines          =   true,
	breakatwhitespace   =   true,
    columns         =   fixed,
	tabsize             =   3
}

\lstdefinestyle{C++}{
    language        =   C, % 语言选C
    basicstyle      =   \zihao{-5}\ttfamily,
    numberstyle     =   \zihao{-5}\ttfamily,
    keywordstyle    =   \bfseries\color{green!40!black},
	commentstyle    =   \itshape\color{purple!40!black},
	identifierstyle =   \color{blue},
	stringstyle     =   \color{orange},
    xleftmargin     =   \parindent,
    belowcaptionskip=   1\baselineskip,
    breaklines      =   true,   % 自动换行,建议不要写太长的行
    columns         =   fixed,  % 如果不加这一句,字间距就不固定,很丑,必须加
    basewidth       =   0.5em,
}

\lstdefinestyle{Lua}{
    language        =   C, % 语言选C++
    basicstyle      =   \zihao{-5}\ttfamily,
    numberstyle     =   \zihao{-5}\ttfamily,
    keywordstyle    =   \bfseries\color{green!40!black},
	commentstyle    =   \itshape\color{purple!40!black},
	identifierstyle =   \color{blue},
	stringstyle     =   \color{orange},
    xleftmargin     =   \parindent,
    belowcaptionskip=   1\baselineskip,
    breaklines      =   true,   % 自动换行,建议不要写太长的行
    columns         =   fixed,  % 如果不加这一句,字间距就不固定,很丑,必须加
    basewidth       =   0.5em,
}

% \lstinputlisting[
%     style       =   C,
%     caption     =   {\bf ff.py},
%     label       =   {ff.py}
% ]{../ff.py}
