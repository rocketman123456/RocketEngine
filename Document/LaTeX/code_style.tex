\graphicspath{ {./} }

\definecolor{dkgreen}{rgb}{0,0.6,0}
\definecolor{dkblue}{rgb}{0,0,0.6}
\definecolor{gray}{rgb}{0.5,0.5,0.5}
\definecolor{mauve}{rgb}{0.58,0,0.82}

\lstset{
    backgroundcolor =   \color{white},    % choose the background color
    tabsize         =   4,
    breaklines      =   true,             % automatic line breaking only at whitespace
    captionpos      =   b,                % sets the caption-position to bottom
    basicstyle      =   \footnotesize\ttfamily, % size of fonts used for the code
    commentstyle    =   \color{mygreen}\rmfamily\itshape, % comment style
    keywordstyle    =   \color{blue},     % keyword style
    stringstyle     =   \color{mymauve}\ttfamily, % string literal style
    numberstyle     =   \ttfamily,
    identifierstyle =   \color{red},
    rulesepcolor    =   \color{red!20!green!20!blue!20},
    escapeinside    =   {\%*}{*)},        % if you want to add LaTeX within your code
    frame           =   lrtb,             % single,
    columns         =   fullflexible,
    %columns         =   fixed,
    language        =   c++,
    flexiblecolumns,
}

\lstdefinestyle{CustomPython}{
	frame               =   l,
	language            =   Python,
	aboveskip           =   3mm,
	belowskip           =   3mm,
	showstringspaces    =   false,
	columns             =   flexible,
	numberstyle         =   \small\color{red},
	basicstyle          =   {\small\ttfamily},
	keywordstyle        =   \color{dkblue},
	commentstyle        =   \color{dkgreen},
	stringstyle         =   \color{mauve},
	breaklines          =   true,
	breakatwhitespace   =   true,
    columns             =   fixed,
	tabsize             =   3,
    numbers             =   left,
    frame               =   lrtb,
}

\lstdefinestyle{C++}{
    language        =   C, % 语言选C
    basicstyle      =   \zihao{-5}\ttfamily,
    numberstyle     =   \zihao{-5}\ttfamily,
    keywordstyle    =   \bfseries\color{green!40!black},
	commentstyle    =   \itshape\color{purple!40!black},
	identifierstyle =   \color{dkblue},
	stringstyle     =   \color{orange},
    xleftmargin     =   \parindent,
    belowcaptionskip=   1\baselineskip,
    breaklines      =   true,   % 自动换行,建议不要写太长的行
    columns         =   fixed,  % 如果不加这一句,字间距就不固定,很丑,必须加
    basewidth       =   0.5em,
    numbers         =   left,
    frame           =   lrtb,
}

\lstdefinestyle{Lua}{
    language        =   C++, % 语言选C++
    basicstyle      =   \zihao{-5}\ttfamily,
    numberstyle     =   \zihao{-5}\ttfamily,
    keywordstyle    =   \bfseries\color{green!40!black},
	commentstyle    =   \itshape\color{purple!40!black},
	identifierstyle =   \color{gray},
	stringstyle     =   \color{orange},
    xleftmargin     =   \parindent,
    belowcaptionskip=   1\baselineskip,
    breaklines      =   true,   % 自动换行,建议不要写太长的行
    columns         =   fixed,  % 如果不加这一句,字间距就不固定,很丑,必须加
    basewidth       =   0.5em,
    numbers         =   left,
    frame           =   lrtb,
}

% Usage
% \lstinputlisting[
%     style       =   C,
%     caption     =   {\bf ff.py},
%     label       =   {ff.py}
% ]{../ff.py}

\makeatletter
\newenvironment{breakablealgorithm} 
    {% \begin{breakablealgorithm}
    \begin{center}
        \refstepcounter{algorithm}% New algorithm
        \hrule height.8pt depth0pt \kern2pt% \@fs@pre for \@fs@ruled
        \renewcommand{\caption}[2][\relax]{% Make a new \caption
        {\raggedright\textbf{\ALG@name~\thealgorithm} ##2\par}%
            \ifx\relax##1\relax % #1 is \relax
                \addcontentsline{loa}{algorithm}{\protect\numberline{\thealgorithm}##2}%
            \else % #1 is not \relax
                \addcontentsline{loa}{algorithm}{\protect\numberline{\thealgorithm}##1}%
            \fi
            \kern2pt\hrule\kern2pt
        }
    }{% \end{breakablealgorithm}
        \kern2pt\hrule\relax% \@fs@post for \@fs@ruled
    \end{center}
    }
\makeatother
