\section{打开第一个窗口}
当然作为一个3D程序,需要通过显示器才能展示我们的工作成果,
而在现代操作系统中,必须打开窗口才能开始展示东西,就和其他软件一样。
为了保证本项目的跨平台能力,需要尽可能一致的为不同平台提供相同的封装,
在这里本项目选择了常用的GLFW库\footnote{\nolinkurl{https://www.glfw.org/}}。\\

暂时本项目只使用OpenGL进行渲染,根据官网的教程,可以很容易写出第一个窗口程序:

\begin{lstlisting}[style=C++,title={Engine/glfw.cpp}]
#include "Memory/MemoryCheck.h"

#define GLFW_INCLUDE_NONE
#include <GLFW/glfw3.h>
#include <glad/glad.h>

#include <iostream>

void processInput(GLFWwindow *window);

int main() {
	glfwInit();
    glfwWindowHint(GLFW_CONTEXT_VERSION_MAJOR, 3);
    glfwWindowHint(GLFW_CONTEXT_VERSION_MINOR, 3);
    glfwWindowHint(GLFW_OPENGL_PROFILE, GLFW_OPENGL_CORE_PROFILE);
	glfwWindowHint(GLFW_RESIZABLE, false);
#if defined(RK_MACOS)
    glfwWindowHint(GLFW_OPENGL_FORWARD_COMPAT, GL_TRUE);
#endif

	GLFWwindow* window = glfwCreateWindow(1280, 720, "Rocket", NULL, NULL);
	if (window == NULL) {
		std::cout << "Failed to create GLFW window" << std::endl;
		glfwTerminate();
		return -1;
	}
	glfwMakeContextCurrent(window);

	if (!gladLoadGLLoader((GLADloadproc)glfwGetProcAddress)) {
        std::cout << "Failed to initialize GLAD" << std::endl;
        return -1;
    }

	while (!glfwWindowShouldClose(window)) {
        processInput(window);

        glClearColor(0.2f, 0.3f, 0.3f, 1.0f);
        glClear(GL_COLOR_BUFFER_BIT);

        glfwSwapBuffers(window);
        glfwPollEvents();
    }

	glfwDestroyWindow(window);
	glfwTerminate();

    return 0;
}

void processInput(GLFWwindow *window)
{
    if(glfwGetKey(window, GLFW_KEY_ESCAPE) == GLFW_PRESS)
        glfwSetWindowShouldClose(window, true);
}
\end{lstlisting}

之后开展的渲染学习项目,将从这段程序开始扩展,并接入软渲染器来实现场景初步渲染的能力。
