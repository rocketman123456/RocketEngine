\section{内存检测模块}

作为第一个编写的模块,我选择了内存检测的模块。该模块的目的是提供一个跨平台,线程安全的内存泄漏检测器。
参考了Vorbrodt的Blog\footnote{\nolinkurl{https://vorbrodt.blog/2021/05/27/how-to-detect-memory-leaks/}},
对其添加了线程安全锁后,实现了一个小型的内存检测器。
由于通过宏定义完成了内存检测器的行数输出,所以该检测器只能用于普通的new和delete,
对于placement new就无力处理了。若想要不用宏定义也可以获取代码调用位置,
可能就只能等待C++20的std::source\_location的功能了。\\

该内存检测模块主要通过检测new与delete的配对问题来检测内存泄漏。
一次new对应一次delete,一次new[]对应delete[],避免错配。
该模块使用了set来存储信息,在new的过程中插入信息,delete的过程中检测并删除对应new的信息,达到匹配的效果。
为了能够在程序退出时自动输出内存泄漏检查信息,定义了一个$dump\_all$的全局变量,
当程序退出时,该变量自动析构,执行最后的内存泄漏检测并输出结果。\\

\lstinputlisting[
    style=C++,title={MemoryCheck.h} \label{chapter01/code/MemoryCheck.h}
]{chapter01/code/MemoryCheck.h}

\lstinputlisting[
    style=C++,title={MemoryCheck.cpp} \label{chapter01/code/MemoryCheck.cpp}
]{chapter01/code/MemoryCheck.cpp}

关于内存检测模块,后续仍有很多改进空间,首先需要的就是移除宏定义对new和delete的限制,
使其能够运行调用placement的new与delete版本,以及可以在此添加自定义的内存管理器,进行自定义内存分配策略。
在新的C++中,已经有内存管理器的相关内容,在未来的看法中,可以考虑基于C++标准库来实现内存管理的功能。\\

除此之外,还可以通过重载new与delete函数,来调用别的底层内存分配库,比如在这里利用了微软推出的mi\_malloc库,相比系统自带的malloc函数,可以提供更高的性能。
不过需要注意的是,在不同的模块间,若采用了不同的底层内存库,有可能导致程序运行时出差,难以排查,这点需要额外注意。

\lstinputlisting[
    style=C++,title={MemoryDefine.h} \label{chapter01/code/MemoryDefine.h}
]{chapter01/code/MemoryDefine.h}

\lstinputlisting[
    style=C++,title={MemoryDefine.cpp} \label{chapter01/code/MemoryDefine.cpp}
]{chapter01/code/MemoryDefine.cpp}

% \begin{breakablealgorithm}
% \caption{Init Records} \label{initialize function}
% \begin{algorithmic}
%     \Procedure{Init Records}{$size$}
%         \State Initialize Static \emph{MemoryRecord Set}
%         \State Initialize Static \emph{DeleteTypeMismatch List}
%         \State Initialize Static \emph{MultiDelete List}
% \EndProcedure
% \end{algorithmic}
% \end{breakablealgorithm}

% \begin{breakablealgorithm}
% \caption{New} \label{new function}
% \begin{algorithmic}
% \Procedure{new}{$size$}                  \Comment{new function}
%     \State malloc a Memory Block
%     \State Insert a Memory Block Info in \emph{MemoryRecord Set}
%     \State return Memory Block Pointer
% \EndProcedure
% \end{algorithmic}
% \end{breakablealgorithm}

% \begin{breakablealgorithm}
% \caption{Delete} \label{delete function}
% \begin{algorithmic}                     \Comment{delete function}
% \Procedure{delete}{$ptr$}
%     \State result = Find $ptr$ in \emph{Memory Record Set}
%     \If{$result = True$}
%         \If{$delete type match$}
%             \State Remove Pointer Info in \emph{MemoryRecord Set}
%         \Else
%             \State Record Info in \emph{DeleteTypeMismatch List}
%         \EndIf
%     \Else
%         \State Record Info in \emph{MultiDelete List}
%     \EndIf
% \EndProcedure
% \end{algorithmic}
% \end{breakablealgorithm}

% \begin{breakablealgorithm}
% \caption{ExitCheck} \label{exit check}
% \begin{algorithmic}
% \Procedure{ExitCheck}{$ptr$}
%         \State Find Leak in \emph{MemoryRecord Set} and Dump Info
%         \State Find Mismatch in \emph{DeleteTypeMismatch List} and Dump Info
%         \State Find Mismatch in \emph{MultiDelete List} and Dump Info
% \EndProcedure
% \end{algorithmic}
% \end{breakablealgorithm}
