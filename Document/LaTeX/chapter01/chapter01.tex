\maketitle
\chapter{从零开始}

\section{第一个简单项目}
作为笔记的第一部分,本节主要介绍从零开始搭建一个小型项目的过程。
大部分的C++项目都需要选择(虽然其他项目也是,但是没有C++这么多的选项)一个基本的开发环境,而CMake是其中一个流行的选择。
从github看去,很多国内外大型的开源项目都选择了CMake进行项目的配置和管理。前几年我也是CMake的推荐者,
但学习过CMake的我仍然觉得CMake过于死板,需要学习额外的DSL(Domain Specific Language),无法轻易更改编译过程。
经过查找,我发现了一个国产的开源项目xmake[],可以用Lua编写项目配置文件,符合C++编译过程的直觉,且可以通过xmake避免复杂繁琐的外部库安装配置过程,让人感觉很舒心。
最终本项目决定使用xmake作为项目配置软件。\\

当然,本项目从熟悉的Hello World开始,一步步搭建起项目框架。首先是函数主体:\\

% \FloatBarrier
\begin{lstlisting}[style=C++,title={Engine/main.cpp}]
#include <iostream>

int main() {
    std::cout << "Hello World" << std::endl;
    return 0;
}
\end{lstlisting}
% \FloatBarrier

% \lstinputlisting[style=C++,title={main.cpp}]{chapter01/code/main.cpp}
% \lstinputlisting[style=Lua,title={xmake.lua}]{chapter01/code/xmake.lua}

这个就是简单的标准的Hello World,通过xmake可以简单的配置项目:\\

% \FloatBarrier
\begin{lstlisting}[style=Lua,title={xmake.lua}]
target("hello_world")
    set_kind("binary")
    add_files("Engine/*.cpp")
\end{lstlisting}
% \FloatBarrier

在命令行输入