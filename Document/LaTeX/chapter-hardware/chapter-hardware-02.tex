\section{虚拟机实现}

虚拟机的执行主要分为以下几步:

1.从PC寄存器加载指令

2.对PC寄存器加一

3.根据加载的指令决定执行命令

4.执行命令

5.回到第一步

指令的执行在虚拟机里使用switch case来决定执行程序。这个虚拟机的运行流程十分简单,具体实现代码如下:

% \lstinputlisting[
%     style=C++,title={lc3vm.h} \label{chapter-hardware/code/lc3vm.h}
% ]{chapter-hardware/code/lc3vm.h}

% \lstinputlisting[
%     style=C++,title={lc3vm.c} \label{chapter-hardware/code/lc3vm.c}
% ]{chapter-hardware/code/lc3vm.c}

% \lstinputlisting[
%     style=C++,title={lc3.c} \label{chapter-hardware/code/lc3.c}
% ]{chapter-hardware/code/lc3.c}

\lstinputlisting[
    style=C++,title={lc3vm-basic.c} \label{chapter-hardware/code/lc3vm-basic.c}
]{chapter-hardware/code/lc3vm-basic.c}

这个虚拟机还有高级版本,可以加载一个微型的系统,代替c的switch case,直接执行程序代码。
