\section{LC3电脑结构简介}

LC3的电脑结构\footnote{\nolinkurl{http://users.ece.utexas.edu/~patt/07s.360N/handouts/360n.appC.pdf}}较为简单,
它的指令集模仿了x86的指令集,是一个精简版的指令集,但是这个LC3电脑拥有现代CPU的所有基本概念,非常适合用于初学者学习研究。

LC3的最大内存为65536个uint16\_t,它的最大内存空间为128KiB,虽然很小,但是足够运行一些示例程序。
当然,作为一个虚拟机,该内存空间也可以扩展,加载更大与更复杂的程序。在虚拟机中,内存使用uint16\_t的数组来表示。

在LC3中,它的CPU通过寄存器来操作数据,执行指定指令。LC3一共有10个寄存器,分别是R0-R7一共8个通用寄存器,一个PC指令寄存器和一个COND标志位寄存器。
通用寄存器可以进行各种程序计算,PC寄存器指向下一个待执行的指令,COND寄存器保存之前计算的结果标志。寄存器在虚拟机中同样由数组表示。

\subsection{LC3指令集}

指令用于告诉CPU应该执行什么样的操作,以及该操作附带的参数。
在LC3中仅有16个指令,每个指令16bit长,左4位用于表示指令代码,其余位数用于存储指令参数。
这16个指令分别为:OP\_BR、OP\_ADD、OP\_LD、OP\_ST、OP\_JSR、OP\_AND、OP\_LDR、OP\_STR、OP\_RTI、OP\_NOT、OP\_LDI、OP\_STI、OP\_JMP、OP\_RES、OP\_LEA、OP\_TRAP。
由于LC3的指令数量很少,该指令集属于RISC指令集\footnote{\nolinkurl{https://en.wikipedia.org/wiki/Reduced_instruction_set_computer}}。

\subsection{标志位}

LC3的标志位仅有三个符号:FL\_POS、FL\_ZRO、FL\_NEG。该符号用于表示逻辑表达式的比较结果。
