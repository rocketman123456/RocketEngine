为了深入了解计算机的运行原理,加深对程序编译与运行流程的的理解,
本章主要介绍构造LC3 Computer\footnote{\nolinkurl{https://en.wikipedia.org/wiki/Little_Computer_3}}的虚拟机,
并构建一个相对简单的C程序编译器,来生成LC3运行的汇编代码,最后通过汇编代码生成二进制文件,在虚拟机中运行。
这个虚拟机可以运行运行比较简单的程序,也可以别人编写的2048小程序\footnote{\nolinkurl{https://github.com/rpendleton/lc3-2048}}。

使用LC3作为计算机原理展示的例子,不仅仅是因为这个出现在了Introduction to Computing Systems: From Bits and Gates to C and Beyond, 2nd Edition\footnote{\nolinkurl{https://highered.mheducation.com/sites/0072467509/index.html}}一书中,
也因为这个系统结构简单,且其运行的指令集与当前电脑运行的指令集有很多相似的地方,可以由此抛砖引玉,沿着这个方向继续深入学习。

本章的内容主要参考以下几个资料\footnote{\nolinkurl{https://justinmeiners.github.io/lc3-vm/}}。
