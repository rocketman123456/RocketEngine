\section{日志输出}

\subsection{为什么需要日志输出}
在程序运行过程中,除了编译器带给我们的单步调试功能,开发过程中也需要额外的信息输出方式,
用于快速检查程序输出内容是否正常以及快速定位可能存在bug的代码位置。
比如输出log日志可以快速检测各个模块初始化是否按照预定顺序执行,是否全部成功初始化;
在调用函数的过程中,可以检测运行逻辑是否与预期一样。
更重要的是,在发布出去的软件中,我们对其进行错误追踪的方式可能只有日志输出一个渠道,
所以有一个完善的日志输出模块,对我们的开发工作会有很大帮助。
为了达到这个目的,本节主要介绍日志输出的部分。

\subsection{封装日志模块}
日志输出模块首先需要保证系统性能,其次可以按照不同的日志等级输出信息,便于分类,再次则需要保证多线程输出信息的安全,
最后则是能够输出到日志文件等功能。经过挑选,本项目采用了spdlog作为日志输出的底层库,并对其进行了简单的封装。
主要包括$Log.h^{\ref{chapter-start/code/Log.h}}$与$Log.cpp^{\ref{chapter-start/code/Log.cpp}}$两个文件。
在程序运行开始需要初始化Log,在项目终止时需要终止日志输出(虽然可以依靠自动析构,但是显式调用可以使流程更加清晰)。
利用宏定义,可以快速添加新的日志分类。

\lstinputlisting[
    style=C++,title={Log.h} \label{chapter-start/code/Log.h}
]{chapter-start/code/Log.h}

\lstinputlisting[
    style=C++,title={Log.h} \label{chapter-start/code/Log.cpp}
]{chapter-start/code/Log.cpp}
