\section{打开第一个窗口}
当然作为一个3D程序,需要通过显示器才能展示我们的工作成果。不同的系统有不同的显示API,为了保证本项目的跨平台能力,
需要尽可能一致的为不同平台提供相同的封装,在这里本项目选择了常用的GLFW库\footnote{\nolinkurl{https://www.glfw.org/}}。

本项目有两套渲染器,一个是Vulkan渲染器,一个是软渲染器。软渲染器在CPU中渲染完图片后,使用OpenGL进行屏幕渲染输出,
该软渲染器主要参考了GAMES102课程的代码\footnote{\nolinkurl{https://games-cn.org/games102/}}。
根据LearnOpenGL教程\footnote{\nolinkurl{https://learnopengl-cn.github.io/}},可以很容易写出第一个窗口程序:

\lstinputlisting[
    style=C++,title={glfw.cpp} \label{chapter-start/code/glfw.cpp}
]{chapter-start/code/glfw.cpp}

之后开展的渲染学习项目,将从这段程序开始扩展,并接入软渲染器来实现场景初步渲染的能力。
