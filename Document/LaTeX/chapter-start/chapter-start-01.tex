\section{第一个简单项目}
作为笔记的第一部分,本节主要介绍从零开始搭建一个小型项目的过程。
大部分的C++项目都需要选择一个基本的编译配置,而CMake是其中一个流行的选择。
从github看去,很多国内外大型的开源项目都选择了CMake进行项目的配置和管理。前几年我也是CMake的推荐者,
但学习过CMake的我仍然觉得CMake过于死板,需要学习额外的DSL(Domain Specific Language),学习曲线比较陡峭。
经过查找,我发现了一个国产的开源项目xmake\footnote{\nolinkurl{https://xmake.io}},可以用Lua编写项目配置文件,符合C++编译过程的直觉,且可以通过xmake避免复杂繁琐的外部库安装配置过程,让人感觉很舒心。
最终本项目决定使用xmake作为项目配置软件。\\

当然,本项目从熟悉的Hello World开始,一步步搭建起项目框架。首先是函数主体:\\

\lstinputlisting[
    style=C++,title={main.cpp} \label{chapter-start/code/main.cpp}
]{chapter-start/code/main.cpp}

这个就是简单的标准的Hello World,通过xmake可以简单的配置项目:\\

\lstinputlisting[
    style=Lua,title={xmake.lua} \label{chapter-start/code/xmake.lua}
]{chapter-start/code/xmake.lua}

在命令行输入\emph{xmake}后,就可以自动编译项目,输入\emph{xmake r hello\_world}后,就可以运行对应项目。
通过这种流程,可以进行快速的项目编译与运行测试。\\

在后续章节中,若是对应的模块代码过于冗长,我将使用伪代码的形式提供运行的主要流程,对于的代码可以在github仓库中查看。
